\begin{flushleft}
	\Large
	\textbf{�berblick\\}
	\vspace{1.5cm}
	
	\large
Das Ziel von Sombrero ist die Erstellung eines quelloffenen Werkzeugs zur Verwaltung von KNX. Mit Verwaltung ist hier nicht die Konfiguration der KNX-Ger�te selbst gemeint, sondern die Kontrolle derselben in einer strukturierten und benutzerfreundlichen Umgebung.

Um eine einfache Benutzerschnittstelle zu erm�glichen, wurde entschieden, ein Webservice als Frontend zu verwenden, auf das man mit jedem Webbrowser zugreifen kann. Um es den Endbenutzern noch einfacher zu machen, sich an die neue Umgebung zu gew�hnen, wurde versucht, so stark wie m�glich auf bestehenden Erwartungen aufzubauen. Ein Kernthema hierbei war die sinnvolle Einbindung der Maus, besonders von Linksklicks und Drag \& Drop. Weil die Endbenutzer von Kontrollelementen zur Verwaltung verwirrt werden k�nnten, wurde eine Benutzerverwaltung eingef�hrt und die administrativen Aufgaben speziellen Administratorbenutzern vorbehalten.

Um den Entwicklungsprozess so einfach und schnell wie m�glich zu gestalten, schlugen wir einen agilen und dynamischen Pfad im Hinblick auf das Projektmanagement ein und verwendeten eine modifizierte Version des "Getting Real" Prozessmodells. Um die Entwicklungszeit und Fehleranf�lligkeit zu minimieren, wurden die Programmiersprache Scala, das JavaScript Framework JQuery und das Webframework Lift verwendet.
\end{flushleft}
