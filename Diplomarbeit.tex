\documentclass[english, a4paper, 12pt]{report}

%\usepackage[english]{babel}					%Deutsche Silbentrennung etc.
\usepackage[USenglish]{babel}
\usepackage[latin1]{inputenc}	%Umlaute in .tex Files normal schreibbar
\usepackage{helvet}						%Helvetic als Schriftart
\usepackage{courier}					%Courier als Schriftart f�r Listings
\usepackage{fancyhdr}					%Kopf- und Fu�zeilen �ndern
\usepackage{a4}								%A4 Randeinstellungen
\usepackage{makeidx}					%Indexkommandos
\usepackage{listings}					%F�r zeilennummerierte Listings mit Hintergrund
\usepackage{color}						%F�r grauen Hintergrund in Listings
\usepackage{setspace}					%Gr��erer Zeilenabstand
\usepackage{graphicx}					%Grafiken einbinden
\usepackage{sectsty}					%Format der �berschriften um�ndern
\usepackage{hyperref}
%Darstellung des Glossars und Abk�rzungsverzeichnisses einstellen
\usepackage[style=altlist, hypertoc=true, hyper=true, number=none, acronym=true]{glossary}
\setacronymnamefmt{gloshort}
\makeacronym
\makeglossary

%Dokumentationen zu den Paketen finden sich im Installationsordner
%(normalerweise C:\Programme\texmf) unter docs und dort auch im Unterverzeichnis latex.

%Das Kompilieren des Dokuments ben�tigt bis zu 3 Durchl�ufe im alle Referenzen und
%Literatureintr�ge korrekt einzubinden.


%Definition des Aussehens der Listings
\definecolor{listinggray}{gray}{1.0}
\lstset{
	language=Java,
	frame=ltrb
	backgroundcolor=\color{listinggray},
	%basicstyle=\linespread{1.0}\ttfamily\small,
	basicstyle=\linespread{1.0}\ttfamily\small,
	commentstyle=\textit,
	tabsize=2,
	float=ph,
	extendedchars,
	breaklines,
	prebreak={\space\hbox{\ensuremath\hookleftarrow}},,
	numbers=left,
	numberstyle=\small,
	stringstyle=\textsl,
	showstringspaces=false,
	captionpos=b,
	aboveskip=16pt
}
%Eigene Kommandos
\newcommand{\zb}{z.B.\ }															 %z.B.
\newcommand{\tm}{\texttrademark \ }										%TM - Zeichen
\newcommand{\sk}[1]{\emph{siehe Kapitel \ref{#1}}}		%siehe Kapitel <Referenz>
\newcommand{\lil}[1]{\emph{Listing \ref{#1}}}					%Listing <Referenz>
\newcommand{\slil}[1]{\emph{siehe Listing \ref{#1}}}	%siehe Listing <Referenz>
\newcommand{\pr}{$\rightarrow\ $}											%Pfeil nach rechts

\graphicspath{{images/}}

%Es soll ein Index f�r diese Diplomarbeit erzeugt werden
\makeindex

%L�ngen- und Absatzeinstellungen
\parindent=0pt		%Kein Einr�cken der ersten Zeile eines Absatzes
\parskip=12pt			%12pt Abstand zwischen 2 Abs�tzen
%\doublespacing 	 %Doppelter Zeilenabstand
\onehalfspacing	  %Eineinhalbfacher Zeilenabstand

\setlength{\headheight}{15pt}		%Kopfzeile vergr��ern (wegen 12pt Schriftgr��e)	
\addtolength{\textwidth}{1.5cm}	%Rechten Rand verkleinern


%--------------------------------------------------------------------------
% Beginn Dokument
%--------------------------------------------------------------------------


\begin{document}
	\sffamily											%Schriftart setzen
	\allsectionsfont{\sffamily}		%Schrift f�r �berschrift setzen
	
	\input{title}									%Externe .tex Datei f�r Titel einbinden
	\clearpage										%Neue Seite beginnen

	\pagestyle{plain}							%Nur Fu�zeile mit Seitennummer anzeigen lassen
	\pagenumbering{roman}					%R�mische Nummerierung vor der eigentlichen Diplomarbeit
	\setcounter{page}{1}					%Bei 1 mit Nummerierung beginnen

	\addcontentsline{toc}{chapter}{Eidesstattliche Erkl�rung}	%Erkl�rung h�ndisch ins Inhaltsverzeichnis einf�gen
	\begin{flushleft}
	\Large
	\textbf{Eidesstattliche Erkl�rung\\}
	\vspace{1.5cm}

	\large Ich erkl�re an Eides statt, dass ich die vorliegende Diplomarbeit selbst�ndig und ohne fremde Hilfe verfasst, andere als die angegebenen Quellen und Hilfsmittel nicht benutzt und die den benutzten Quellen w�rtlich und inhaltlich entnommenen Stellen als solche erkenntlich gemacht habe. \\

	\vspace{1,5cm}
	Alexander C. Steiner\\
	\vspace{1cm}
	\rule{400pt}{1pt}

	\vspace{1,5cm}
	Gabriel A. Grill\\
	\vspace{1cm}
	\rule{400pt}{1pt}
\end{flushleft}
												%Externe .tex Datei f�r Erkl�rung einf�gen
	\clearpage																%Neue Seite beginnen
	
	
	\addcontentsline{toc}{chapter}{Acknowledgements}
	\begin{flushleft}
	\Large
	\textbf{Acknowledgements\\}
	\vspace{1.5cm}

	\large
		First, we would like to thank our supervisor Harald Haberstroh and our client at the University of Technology, Wolfgang Kastner. Without you, this would not have been possible, and not nearly as interesting. Especially, we thank Harald for putting up with us using not a single piece of technology he had known before.

	Of course, many thanks also go to our families. Without your support, it would have been impossible for us to stay the course for the entire time. We also deeply appreciate your corrections and helpful tips.

	Lastly, we would also like to thank our entire class. This were by far our most fun years at school, and we don't think there has ever been another class with nearly the same attitude as our's.
\end{flushleft}

	\clearpage
	
	\addcontentsline{toc}{chapter}{Abstract}
	\begin{flushleft}
	\Large
	\textbf{Abstract\\}
	\vspace{1.5cm}

	\large
The main goal behind Sombrero is the creation of an open source KNX management tool. Management in this sense is not the configuration of KNX devices, but rather control of them in a structured and user-friendly way.

To present users with a simple interface, it was decided to use a web service, accessible through a web browser, as a front-end. To further ease the adoption through end users, experiences with standard software, most notably heavy use of mouse clicks and drag \& drop, should be applicable in the web interface. To avoid confusion of the end user by management control elements, user management and designated administrator users with extended control capabilities were added.

In order to speed up development, an agile and dynamic approach to project management was taken, and an adopted version of the Getting Real process model was used. To minimize implementation time and error-proneness, the programming language Scala, the JavaScript framework JQuery and the web framework Lift were used.\ref{Iterations}
\end{flushleft}

	\clearpage
    \input{abstract-german}
	\clearpage

	\normalsize

	\markright{INDEX}
	\addcontentsline{toc}{chapter}{Index}
	\tableofcontents
	\clearpage
		
	%Kopfzeilen definieren
	\pagestyle{fancyplain}

	\renewcommand{\sectionmark}[1]{\markright{\thesection\ #1}}
	\renewcommand{\chaptermark}[1]{\markright{\thechapter\ #1}}
	 \lhead[\fancyplain{}{\sffamily\sl\thepage}]{\fancyplain{}{\sffamily\sl\rightmark}}
	 \rhead[\fancyplain{}{\sffamily\sl\leftmark}]{\fancyplain{}{\sffamily\sl\thepage}}
	\cfoot{}

	\pagenumbering{arabic}	%Seiten wieder normal nummerieren
	\setcounter{page}{1}		%Bei 1 beginnen

	%--------------------------------------------------------------------------
	% Kapitel einf�gen
	%--------------------------------------------------------------------------

    \input{chapter-allocation}
	\clearpage
    \input{introduction}
	\clearpage
    \chapter{Project Progression}
%\chapter{Project Progression}
\section{Methodology}
\subsection{Overview}

As we started to plan our project sombrero, we decided to do it in a fresh, modern and agile way, because after 4 years of programming in only imperative and boilerplate stuffed languages with project planning methods like RUP, we thought it would be nice to try something different. So we chose Scala as a lightweight and mostly boilerplate free language for our server side programming, but without a framework this would have been too much work. Luckily there is a web framework in Scala and it's called Lift. It borrows its ideas and concepts from the other most popular frameworks out there. Probably the most popular one is Ruby on Rails. It's created by 37 signals and used by many well known companies like Xing and Twitter.

Because of the methodology's popularity 37 signals wrote a book on how to use it properly in a project, but to make it more adaptable for different situations it was written very abstract. So you could apply it for almost any project. It's called Getting Real. The agile manifest was used as a model to create it. To use this knowledge Gabriel wrote a short adaption of Getting Real for our project. So with this kind of methodology to lead our way we were able to organize our project a fresh and agile way.

  \clearpage
\subsection{Agile Manifesto}
 ``The modern definition of agile software development evolved in the 1990s as part of a reaction against "heavyweight" methods, perceived to be typified by a heavily regulated, regimented, micro-managed use of the waterfall model of development. The processes originating from this use of the waterfall model were seen as bureaucratic, slow, demeaning, and inconsistent with the ways in which software developers actually perform effective work.''\cite{wikipedia.org:agile}

 The Agile Manifesto is published in 2001.

 ``The Principals are:
  \begin{itemize}
    \item \textbf{Individuals and interactions} over processes and tools
    \item \textbf{Working software} over comprehensive documentation
    \item \textbf{Customer collaboration} over contract negotiation
    \item \textbf{Responding to change} over following a plan''\cite{agilemanifesto.org}
  \end{itemize}

\subsection{Getting Real}
    ``Getting Real is a smaller, faster, better way to build software. It is about iterations and lowering the cost of change. Getting Real is all about launching, tweaking, and constantly improving which makes it a perfect approach for web-based software.''\cite{37signals:10}

\subsection{Project Manifest}
    The length of an iteration was 4 weeks, holidays not included. Before the start of an iteration a backlog had to be written and posted to the Project Blog. Communication was done mostly per e-Mail. In special cases, like the end of an iteration, a meeting had to be held. The result of every iteration had to be a functional build. Every work related to sombrero had to be documented on the respective Google Docs\footnote[1]{Google Docs is a free, Web-based word processor offered by Google.} spreadsheet.

    Everything else concerning project management had to be done conforming to the rules of the Agile Manifest and the adoption of Getting Real for sombrero.

\subsection{Google Sites, Docs and Groups}
    As a project management tool we used Google Sites\footnote[2]{Google Sites is a like a wiki with Google Gadgets and Blog.} and Google Docs. On Google Sites we had project blog were we posted the specification of every iteration before the beginning of it. After that we posted the allocation of responsibilities. After the end of each iteration we published the result.

    We used Google Docs as our project log. Both of us had a spreadsheet and in each row we filled the respective date, the work we did and the hours spent in.
\subsection{Pivotaltracker}
    The time management during the documentation work was done by Pivotaltracker, a web based tool that our project supervisor Prof. Harald Haberstroh showed us. We really liked working with it and its structure fits our needs much better than Google Sites.
\clearpage
\input{iterations}
\clearpage
\input{challenges}
\clearpage
\input{project-log}
\clearpage
\input{similar-products}
\clearpage
\input{scala-days}
\clearpage
	\clearpage
    \input{technologies}
	\clearpage
    \input{how-to-use}
	\clearpage
    \input{how-to-hack}
	\clearpage
    \input{ideas-for-the-future}
	\clearpage
    \input{conclusion}
  \clearpage

  \addcontentsline{toc}{chapter}{Appendix}
  \appendix
  
	\input{cv}
	\clearpage
  \chapter{Source Structure}

\section{Packages}

  \begin{figure}[h]
  \centering
  \includegraphics[width=0.80\linewidth]{packages.png}
  %\input{packages}
  \caption{package structure}
  \label{fig:packages}
  \end{figure}

This picture probably needs explanation. The Scala world is located under \lstinline!src/main/scala/!. The templates of HTML world are in \lstinline!src/main/webapp/!, the widget resources are in \lstinline!src/main/resources/toserve/!.

The user accesses Sombrero through a web browser. His request either matches one of the externally accessible templates or gets caught by URL rewriting of one of the location classes in the \lstinline!view! package, which will in turn redirect to a template, but supply it with special snippets. The most notable example for URL rewriting would be the room view. The user's request gets rewritten by the \lstinline!RoomLoc! class in \lstinline!view!, which redirects to \lstinline!room.html!, which in turn uses the \lstinline!roomview! snippet in \lstinline!RoomLoc!. \lstinline!RoomLoc! can then proceed, using information saved from the user's original request, to get widget information from the database interface that is the \lstinline!model! package and use it to construct the widgets that make up the room, using \lstinline!widget!'s various UI classes.

Templates can use other snippets, which don't depend on the user's request, too, these are located in the \lstinline!snippet! package. The favorites bar would be an example of such a snippet. KNX widgets use the KNX network to query the device status upon creation and forward the user's actions to the actual devices. The database is used to save widget positions. The \lstinline!comet! package includes everything that pushes data back to the user. This includes data about status changes in the KNX network, which is translated into UI changes by the widgets. The util package includes utility functionality used by every package (drawing all the arrows would have seriously impacted the clarity of the image), most notably the \lstinline!JavaScriptHelper! and the \lstinline!WidgetList!. Finally, the \lstinline!bootstrap.liftweb! package includes a single class, \lstinline!Boot!, which holds the Lift configuration.


\section{Database Model}

Because we use Lift's Mapper ORM framewrok, all database tables are represented by classes and their companion objects. These classes reside in the \lstinline!model! package. Note that although every Mapper class needs a companion object, they have been omitted in the following diagram for better readability. Many-to-many-relations have been implemented as seperate entities.

  \begin{figure}[h]
  \centering
  \includegraphics[width=0.80\linewidth]{model.png}
  %\input{model}
  \caption{database structure}
  \label{fig:model}
  \end{figure}

The \lstinline!User! table uses Lift's \lstinline!MegaProtoUser! to save user data. This is used to save separate widget positions and favourites per user. The \lstinline!Room! table is used to model the room structure. A room without a parent is considered to be a root room. The \lstinline!Widget! table itself only saves the name of a widget, it is mostly used to provide a single endpoint for all the relations that should lead to the different \lstinline!WidgetData! subclasses. \lstinline!WidgetData! is the trait that needs to be mixed in by all the tables that represent different widget types, their companion objects must mix in \lstinline!WidgetMetaData!. \lstinline!Widget! also has some convenience methods that make it easier to implement new \lstinline!WidgetData! tables. The \lstinline!RoomlinkWidget! and \lstinline!KNXWidget! classes are the \lstinline!WidgetData! implementations for Roomlinks and KNX widgets, respectively. \lstinline!KNXGroup! provides a device to assign multiple KNX addresses to a single KNX widget. Finally, \lstinline!KNXRouter! is a table that contains only a single row, which contains the IP address of the KNX router used to access the local KNX network.

\section{Widget Structure}
To make the widget structure in Sombrero more understandable this diagram was created. It shows the inheritance between the widget Scala classes and it describes the interface between client JavaScript widgets and server Scala classes. All Scala classes are in the placed in Lift box and all JavaScript classes in the JavaScript box.

The JavaScript widgets are all JQuery UI widgets and represent the three types of widgets, which are described in chapter User Mode. These widgets are configurable. This configuration is done by the Scala widget classes. Through this configuration the JQuery UI widgets can be adopted to a special kind of device widget like lamp. It's also possible to use this create a non KNX widget like the room link widget.

During the rendering process is all JQuery UI widget code provided by the Scala classes and on page load are all initialized. The Scala classes now communicate with the JQuery UI widgets through AJAX and Comet. To ease the interface between JavaScript and Scala the class JavaScriptHelper was build.
  \begin{figure}[h]
  \centering
  \includegraphics[width=0.80\linewidth]{widgets.png}
  %\input{widgets}
  \caption{widget structure}
  \label{fig:widgets}
  \end{figure}

	\clearpage
\glossary{ name=Individualsoftware, description={Individualsoftware zeichnet sich dadurch aus, dass sie nur f�r einen oder f�r wenige Anwendungsf�lle geschaffen wird.}}
\clearpage
	
	%--------------------------------------------------------------------------
	% Anhang
	%--------------------------------------------------------------------------

	%\chapter[List of Figures]{}

		\addcontentsline{toc}{chapter}{List of Figures}
	\listoffigures
	\clearpage
	
	%\addcontentsline{toc}{section}{List of Tables}
	%\listoftables
	%\clearpage

	\addcontentsline{toc}{chapter}{List of Listings}
	\lstlistoflistings
	\clearpage
	
    	\addcontentsline{toc}{chapter}{Glossary}
    \printglossary 
	\clearpage


  	\addcontentsline{toc}{chapter}{Index}
  \printindex
  \clearpage

	%\bibliographystyle{alpha} 		 %Standardstyle
	%\bibliographystyle{dinat}		 %Style und Layout nach DIN 1502
	%\bibliography{tiger}					 %Literaturverzeichnis einf�gen
	
	\clearpage
	\addcontentsline{toc}{chapter}{Bibliography}
	\input{bibliography}
    \clearpage
\end{document}
