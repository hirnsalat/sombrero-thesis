%\chapter{Technologies}
\section{KNX}
``More convenience, more safety, higher energy savings: The demand for building management systems is continuously increasing.

Whether in a single-family house or in an office complex, the demand for comfort and versatility in the management of air-conditioning, lighting and access control systems is growing. At the same time, the efficient use of energy is becoming increasingly important. More convenience and safety coupled with lower energy consumption can however only be achieved by intelligent control and monitoring of all products involved. This however implies more wiring, running from the sensors and actuators to the control and monitoring centers. Such a mass of wiring in turn means higher design and installation effort, increased fire risk and soaring costs.

The Answer: KNX /- the Worldwide STANDARD for Home and Building Control''\cite{knx.org}

``KNX is the successor to, and convergence of, three previous standards: the European Home Systems Protocol  (EHS)\footnote[1]{EHS is protocol that's target group were home appliances control and communication using Powerline\footnote[2]{Communication over wires carrying electrical power is called Powerline.}}, BatiBUS\footnote[3]{The BatiBUS is a deprecated field bus.}, and the European Installation Bus (EIB or Instabus)\footnote[4]{EIB describes how sensor and actuators can be connected during installation and how they communicate.}. The KNX standard is administered by the Konnex Association.

KNX defines several physical communication media:

\begin{itemize}
    \item \textbf{Twisted pair wiring (inherited from the BatiBUS and EIB Instabus standards)}
    \item \textbf{Powerline networking (inherited from EIB and EHS - similar to that used by X10\footnote[5]{X10 is a Powerline based home automation protocol.})}
    \item \textbf{Radio (KNX-RF\footnote[6]{KNX Radio Frequency is defined in the KNX specification.})}
    \item \textbf{Infrared}
    \item \textbf{Ethernet (also known as EIBnet/IP or KNXnet/IP)}
\end{itemize}

KNX is designed to be independent of any particular hardware platform. A KNX Device Network can be controlled by anything from an 8-bit microcontroller to a PC, according to the needs of a particular implementation. The most common form of installation is over twisted pair medium.

KNX is approved as an open standard to:

\begin{itemize}
    \item \textbf{International standard (ISO/IEC 14543-3)}
    \item \textbf{Canadian standard (CSA-ISO/IEC 14543-3)}
    \item \textbf{European Standard (CENELEC EN 50090 and CEN EN 13321-1)}
    \item \textbf{China Guo Biao(GB/Z 20965)}
\end{itemize}

KNX has more than 160 members/manufacturers including:

\begin{itemize}
    \item \textbf{ABB}
    \item \textbf{Bosch}
    \item \textbf{Miele \& Cie KG}
    \item \textbf{Schneider Electric Industries S.A.}
    \item \textbf{Siemens}
\end{itemize}

The complete list can be found at \url{http://www.knx.org}.

There are three categories of KNX devices:

\begin{itemize}
    \item \textbf{A-mode or "Automatic mode"}
        devices automatically configure themselves, and are intended to be sold to and installed by the end user.
    \item \textbf{E-mode or "Easy mode"}
        devices require basic training to install. Their behavior is pre-programmed, but they have configuration parameters that need to be tailored to the user's requirements.
    \item \textbf{S-mode or "System mode"}
        devices are used in the creation of bespoke building automation systems. S-mode devices have no default behavior, and must be programmed and installed by specialist technicians.''\cite{wikipedia.org:knx}
\end{itemize}
